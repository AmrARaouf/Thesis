\documentclass[12pt, a4paper, fleqn]{memoir}%makeidx

%******************************************************************************
% STYLE
%******************************************************************************
\input{style.tex}

%******************************************************************************
% BEGIN DOCUMENT
%******************************************************************************
\begin{document}

%******************************************************************************
% FRONT MATTER
%******************************************************************************
\frontmatter

%******************************************************************************
% EMPTY PAGE
%******************************************************************************
\pagestyle{empty}
This is actually the first page of the thesis and will be discarded after the print out. This is done because 
the title page has to be an even page. The memoir style package used by this template makes different indentations 
for odd and even pages which is usally done for better readability.  
\clearpage

%******************************************************************************
% TITLE PAGE
%******************************************************************************
\pagestyle{empty}
\rmfamily
\noindent
\begin{center}
University of Augsburg\\
Faculty of Applied Computer Science\\
Department of Computer Science\\
Bachelor's Program in Computer Science\\
\end{center}
\begin{figure}[h]
\centering
\includegraphics[width=0.25\textwidth]{logo.png}
\end{figure}
\vfill\vfill
\begin{center}
\Large
Bachelor's Thesis\\
\end{center}
\vspace{2.0em}
\begin{center}
\Large
\LARGE Engagement Detection\\ \vspace{10pt} 
\Large Inferring conversational engagement from nonverbal behaviour
\end{center}
\vspace{2.0em}
\begin{center}
    \normalsize
    submitted by\\
    \large
    Amr Abdelraouf\\
    \normalsize
    on 31.7.2014
\end{center}
\vspace{2.0em}
\begin{center}
    \normalsize
    Supervisor:\\ 
    Prof. Dr. Elisabeth Andr\'{e} aus Augsburg
\end{center}
\begin{center}
    \normalsize
    Adviser:\\
    MSc. Tobias Baur
\end{center}
\begin{center}
    \normalsize
    Reviewers:\\
    Prof. Dr. Elisabeth Andr\'{e}\\
\end{center}
\cleardoublepage

%******************************************************************************
% ABSTRACT
%******************************************************************************
\chapter*{Abstract}
This is the place where the \textit{abstract} of your thesis is supposed to be. The abstract is an essential part of a thesis, providing a brief summary of the thesis. Students often do not recognise the importance of the abstract and thus do not spend the required time in order to produce a well defined abstract. You should realize that the abstract is the walking advertisement for your thesis. Any reader's interest in your work stands or falls with the motivation provided by your abstract. A student should know that usually the reviewer of his or her thesis start reading with the abstract and the summary while often just making quick scans over some parts of the main chapters. An abstract is what will and has to be remembered.

%******************************************************************************
% STATEMENT & DECLARATION
%******************************************************************************
\chapter*{Statement and Declaration of Consent}
\vfill
\subsubsection*{\LARGE Statement}
Hereby I confirm that this thesis is my own work and that I have documented all sources used.
\vfill
\begin{flushleft}
Amr Abdelraouf
\end{flushleft}  
\begin{flushright}
Augsburg, 3.7.2014 
\end{flushright}
\vfill
\vfill
\subsubsection*{\LARGE Declaration of Consent}
Herewith I agree that my thesis will be made available through the library of the Computer Science Department.
\vfill
\begin{flushleft}
Amr Abdelraouf
\end{flushleft}  
\begin{flushright}
Augsburg, 3.7.2014 
\end{flushright}
\vfill

%******************************************************************************
% TABLE OF CONTENTS
%******************************************************************************
\cleardoublepage
\rmfamily
\normalfont
\pagenumbering{roman}
\pagestyle{headings}
\tableofcontents


%******************************************************************************
% MAIN MATTER
%******************************************************************************
\mainmatter

%##########################################################
\chapter{Introduction}
\label{chap:Introduction}

\section{Motivation}
\label{sec:Motivation}
This thesis was proposed to help measure the engagement of an interviewee in a job interview situation. Through a simple mock interview the interviewee will be assessed on his/her performence. One of the most important attributes of that performence is whether or not s/he is engaged with and attentive to the interviewer. A simple playback of his/her performence coupled with the measurment of his/her engagement level will easily highlight weak spots in the mock interview.

\section{Objectives}
\label{sec:Objectives}
This thesis aims to measure the engagement levels of an interviewee through non verbal behaviour of said interview. It studies the conversational interaction with the interviewer, the responses to certain commands, his/her behaviour during certain segments of the interview.

\section{Outline}
\label{sec:Outline}
This thesis is divided into three main sections. First it introduces the raw input data that will be used for processing each module of the engagement detection. Then it will explain the main modules used for this thesis in detail. And finally it will explain how the data comes together at the end to produce a level of engagement.

%##########################################################
\chapter{Theoretical Background}
\label{chap:TheoreticalBackground}

\section{Setup}
\label{sec:Section}

\subsection{Subject}
The subject of our expirament is the interviewee in our mock interview. The subject is seated aproximately 30 cm from a screen. The subject is asked to adjust to a number of sensors: a Microsoft Kinekt camera, an SMI Eyetracker and a microphone.

\subsection{Agent}
The agent in our expirament is a reference to the interviewer. The agent is simulated by Scenemaker: a software that is used to create a virtual environment and virtual characters that follow a written script. This script contains sentence utterences, gestures, and commands to wait for a reply from the subject. The main scene used contains two virtual agents named Curtis and Gloria who are standing behind a desk to mimic an office interview. On the left lies a white board that is used as an object in our script.

%%##########################################################
\chapter{Events}
\label{chap:Events}
Events are the backbone of the software workings of this thesis. Raw sensor data are converted to events that can be further processed in the software's pipeline. Furthermore external software send events to our own software over a network. These events can be displayed by themselves as output or can be used as inputs to trigger other events.

\section{Event Structure}
Events are constructs of several attributes:
Time:
The clock signature of when the event was triggered.
Duration:
The time duration of the event.
Ptr:
Meta data about the event.
Type:
The type indicates the nature of the meta data wrapped by the event.

\section{Sensors}
\label{sec:Sensors}

\subsection{Microsoft Kinekt}
Kinekt sensors are used to track the skeletal movements of a subject. However in this module we are mainly interested in the movement of the subject's head. Kinekt is used to detect the prepetual displacement of the subject's head which indicates that s/he is nodding. This triggers an event called \textit{HeadNod}. HeadNod is an event measured every 500 ms and its pointer contains a value from 0 to 1 which represents the probability that the subject is nodding his/her head.

\subsection{SMI Eyetracker}
The SMI Eyetracker is used to pinpoint where the subject is currently looking. Since we are dealing with a virtual agent on a screen we consider the top left corner of the screen as the (0,0) coordinate. Displacement to the right and bottom of the looking point on the screen change the values of the x and y coordinates respectively.

The software defines two main areas on the screen. First is the area of the Agent's face. Second is the area of the board that is present in the environment.

When the subject's looking point falls on the area defined for the agent's face it triggers an event called \textit{SubjectFacialGaze}. SubjectFacialGaze's pointer contains a value of either 0 or 1 indicating whether or not the subject is looking at the agent's face. When the gaze enters the facial area SubjectFacialGaze is triggered with the value 1 indicating that it has started and when the gaze leaves the facial area it is triggered with the value 0 indicating that it is complete.

If the subject's gaze falls in the area of the board the event \textit{SubjectObjectGaze} is triggered. Similar to SubjectFacialGaze, the event carries a value of either 0 or 1 indicating whether or not the subject is looking at the board, value 1 when it starts and value 0 when it ends.

\subsection{Microphone}
A microphone is used to record the verbal utterances produced by the subject. When the microphone detects a voice the event \textit{vad} (which is short for..) is fired. When the voice is first detected the event's pointer carries a value of 1. When the voice activity ends the same event is triggered but with value 0 to indicate that the event is complete.

\section{Scenemaker}
\label{sec:Scenemaker}

%%##########################################################
\chapter{Main Modules}
\label{chap:MainModules}

\section{Mutual Facial Gaze}
\label{sec:MutualFacialGaze}

\section{Directed Gaze}
\label{sec:DirectedGaze}

\section{Adjacency Pair}
\label{sec:AdjacencyPair}

\section{Bachchanneling}
\label{sec:Bachchanneling}

%%##########################################################
\chapter{Bayesian Network}
\label{chap:BayesianNetwork}

%%##########################################################
\chapter{Summery}
\label{Summery}

%******************************************************************************
% BIBLIOGRAPHY
%******************************************************************************
\bibliographystyle{plain}
{\small\bibliography{master}}

%******************************************************************************
% APPENDIX
%******************************************************************************
\appendix
\appendixpage*
\chapter{First Appendix}
\label{app:FirstAppendix}
This is the place where the appendices are supposed to be. Appendices are everything that would just blow up your thesis but are still of some interrest for a reader that wants to get a deeper grasp on the details of your work.

%******************************************************************************
% BACK MATTER
%******************************************************************************
\backmatter

%******************************************************************************
% LIST OF SYMBOLS
%******************************************************************************
%\normalfont
%\clearpage
%\chapter[List of Symbols and Abbreviations]{List of Symbols and Abbreviations}
%\begin{center}
%\small
%\begin{longtable}{lp{3.0in}c}
%\toprule
%\multicolumn{1}{c}{Abbreviation} & \multicolumn{1}{c}{Description}\\ \midrule\addlinespace[2pt] \endhead
%\bottomrule\endfoot
%XML & E\textbf{X}tensible \textbf{M}arkup \textbf{L}anguage \\
%XSD & \textbf{X}ML-\textbf{S}chema-\textbf{D}efinition \\
%SFXML & \textbf{S}cene\textbf{F}low E\textbf{X}tensible \textbf{M}arkup \textbf{L}anguage \\
%SFTXL & \textbf{S}cene\textbf{F}low \textbf{T}extual E\textbf{X}pression \textbf{L}anguage \\
%SCXML & \textbf{S}tate\textbf{C}hart E\textbf{X}tensible \textbf{M}arkup \textbf{L}anguage \\
%DOM & \textbf{D}ocument \textbf{O}bject \textbf{M}odel \\
%LR & \textbf{L}eft to \textbf{R}ightmost derivation \\
%LALR & \textbf{L}ook\textbf{A}head LR\\
%NPC & \textbf{N}on-\textbf{P}erson-\textbf{C}haracter\\
%ABL & \textbf{A} \textbf{B}ehavior \textbf{L}anguage\\
%\end{longtable}
%\end{center}

%******************************************************************************
% LIST OF FIGURES
%******************************************************************************
\normalfont
\clearpage
\listoffigures

%******************************************************************************
% LIST OF TABLES
%******************************************************************************
\normalfont
\clearpage
\listoftables

%******************************************************************************
% LIST OF ALGORITHMS
%******************************************************************************
%\normalfont
\clearpage
\listofalgorithms

%******************************************************************************
% END DOCUMENT
%******************************************************************************
\end{document}
