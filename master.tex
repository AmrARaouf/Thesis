\documentclass[12pt, a4paper, fleqn]{memoir}%makeidx

%******************************************************************************
% STYLE
%******************************************************************************
\input{style.tex}

%******************************************************************************
% BEGIN DOCUMENT
%******************************************************************************
\begin{document}

%******************************************************************************
% FRONT MATTER
%******************************************************************************
\frontmatter

%******************************************************************************
% EMPTY PAGE
%******************************************************************************
\pagestyle{empty}
This is actually the first page of the thesis and will be discarded after the print out. This is done because 
the title page has to be an even page. The memoir style package used by this template makes different indentations 
for odd and even pages which is usally done for better readability.  
\clearpage
%******************************************************************************
% TITLE PAGE
%******************************************************************************
\pagestyle{empty}
\rmfamily
\noindent
\begin{center}
University of Augsburg\\
Faculty of Applied Computer Science\\
Department of Computer Science\\
Bachelor's Program in Computer Science\\
\end{center}
\begin{figure}[h]
\centering
\includegraphics[width=0.25\textwidth]{logo.png}
\end{figure}
\vfill\vfill
\begin{center}
\Large
Bachelor's Thesis\\
\end{center}
\vspace{2.0em}
\begin{center}
\Large
\LARGE Engagement Detection\\ \vspace{10pt} 
\Large Inferring conversational engagement from nonverbal behaviour
\end{center}
\vspace{2.0em}
\begin{center}
    \normalsize
    submitted by\\
    \large
    Amr Abdelraouf\\
    \normalsize
    on 31.7.2014
\end{center}
\vspace{2.0em}
\begin{center}
    \normalsize
    Supervisor:\\ 
    Prof. Dr. Elisabeth Andr\'{e} aus Augsburg
\end{center}
\begin{center}
    \normalsize
    Adviser:\\
    MSc. Tobias Baur
\end{center}
\begin{center}
    \normalsize
    Reviewers:\\
    Prof. Dr. Elisabeth Andr\'{e}\\
\end{center}
\cleardoublepage

%******************************************************************************
% ABSTRACT
%******************************************************************************
\chapter*{Abstract}
This is the place where the \textit{abstract} of your thesis is supposed to be. The abstract is an essential part of a thesis, providing a brief summary of the thesis. Students often do not recognise the importance of the abstract and thus do not spend the required time in order to produce a well defined abstract. You should realize that the abstract is the walking advertisement for your thesis. Any reader's interest in your work stands or falls with the motivation provided by your abstract. A student should know that usually the reviewer of his or her thesis start reading with the abstract and the summary while often just making quick scans over some parts of the main chapters. An abstract is what will and has to be remembered.

%******************************************************************************
% ACKNOWLEDGMENTS
%******************************************************************************
\chapter*{Acknowledgments}
Acknowledgements writing allows an author to tell some words of gratitude to those, who turned out to be rather helpful during your thesis writing process. Of course, acknowledgements are not an integral part of a thesis and if you did all your work on your own, you can omit this part. Writing acknowledgements is not obligatory.

%******************************************************************************
% STATEMENT & DECLARATION
%******************************************************************************
\chapter*{Statement and Declaration of Consent}
\vfill
\subsubsection*{\LARGE Statement}
Hereby I confirm that this thesis is my own work and that I have documented all sources used.
\vfill
\begin{flushleft}
Amr Abdelraouf
\end{flushleft}  
\begin{flushright}
Augsburg, 3.7.2014 
\end{flushright}
\vfill
\vfill
\subsubsection*{\LARGE Declaration of Consent}
Herewith I agree that my thesis will be made available through the library of the Computer Science Department.
\vfill
\begin{flushleft}
Amr Abdelraouf
\end{flushleft}  
\begin{flushright}
Augsburg, 3.7.2014 
\end{flushright}
\vfill

%******************************************************************************
% TABLE OF CONTENTS
%******************************************************************************
\cleardoublepage
\rmfamily
\normalfont
\pagenumbering{roman}
\pagestyle{headings}
\tableofcontents


%******************************************************************************
% MAIN MATTER
%******************************************************************************
\mainmatter

%##########################################################
\chapter{Introduction}
\label{chap:Introduction}

\section{Motivation}
\label{sec:Motivation}

\section{Objectives}
\label{sec:Objectives}

\section{Outline}
\label{sec:Outline}

%##########################################################
\chapter{Theoretical Background}
\label{chap:TheoreticalBackground}

\section{Entenhausen}
\label{sec:Entenhausen}
Figure \ref{fig:intro} shows an image while you can cite a paper with \cite{AmirPnueli1985} or several papers with \cite{ThomasRist2004, Rist2002}.


\section{Section}
\label{sec:Section}

\subsection{PseudoCode}
\label{sec:PseudoCode}

%******************************************************************************
% BIBLIOGRAPHY
%******************************************************************************
\bibliographystyle{plain}
{\small\bibliography{master}}

%******************************************************************************
% APPENDIX
%******************************************************************************
\appendix
\appendixpage*
\chapter{First Appendix}
\label{app:FirstAppendix}
This is the place where the appendices are supposed to be. Appendices are everything that would just blow up your thesis but are still of some interrest for a reader that wants to get a deeper grasp on the details of your work.

%******************************************************************************
% BACK MATTER
%******************************************************************************
\backmatter

%******************************************************************************
% LIST OF SYMBOLS
%******************************************************************************
%\normalfont
%\clearpage
%\chapter[List of Symbols and Abbreviations]{List of Symbols and Abbreviations}
%\begin{center}
%\small
%\begin{longtable}{lp{3.0in}c}
%\toprule
%\multicolumn{1}{c}{Abbreviation} & \multicolumn{1}{c}{Description}\\ \midrule\addlinespace[2pt] \endhead
%\bottomrule\endfoot
%XML & E\textbf{X}tensible \textbf{M}arkup \textbf{L}anguage \\
%XSD & \textbf{X}ML-\textbf{S}chema-\textbf{D}efinition \\
%SFXML & \textbf{S}cene\textbf{F}low E\textbf{X}tensible \textbf{M}arkup \textbf{L}anguage \\
%SFTXL & \textbf{S}cene\textbf{F}low \textbf{T}extual E\textbf{X}pression \textbf{L}anguage \\
%SCXML & \textbf{S}tate\textbf{C}hart E\textbf{X}tensible \textbf{M}arkup \textbf{L}anguage \\
%DOM & \textbf{D}ocument \textbf{O}bject \textbf{M}odel \\
%LR & \textbf{L}eft to \textbf{R}ightmost derivation \\
%LALR & \textbf{L}ook\textbf{A}head LR\\
%NPC & \textbf{N}on-\textbf{P}erson-\textbf{C}haracter\\
%ABL & \textbf{A} \textbf{B}ehavior \textbf{L}anguage\\
%\end{longtable}
%\end{center}

%******************************************************************************
% LIST OF FIGURES
%******************************************************************************
\normalfont
\clearpage
\listoffigures

%******************************************************************************
% LIST OF TABLES
%******************************************************************************
\normalfont
\clearpage
\listoftables

%******************************************************************************
% LIST OF ALGORITHMS
%******************************************************************************
%\normalfont
\clearpage
\listofalgorithms

%******************************************************************************
% END DOCUMENT
%******************************************************************************
\end{document}
